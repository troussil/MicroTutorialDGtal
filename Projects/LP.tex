\documentclass[a4paper, 11pt]{article}

\voffset -0cm
\hoffset 0.0cm
\textheight 23cm
\textwidth 16cm
\topmargin 0.0cm
\oddsidemargin 0.0cm
\evensidemargin 0.0cm

\usepackage{epsfig}  
\usepackage{setspace}
\usepackage{fancyheadings}
\usepackage{amsmath}
\usepackage{amssymb}
\usepackage{graphicx}
\usepackage{url}

%algo
\usepackage[english, linesnumbered, ruled, vlined]{algorithm2e}

\title{}
\author{}
\date{}

\newtheorem{qu}{Question}

\begin{document}

\begin{center}
	\LARGE \textbf{Project: ``Recognition of digital circle''}
\end{center}

\section*{Introduction}

The objective of this project is to implement a convex programming based algorithm
for the recognition of digital circles. 

We expect from you:
\begin{itemize}
\item A short report with answers to the ``formal'' questions and a
  description of your implementation choices and results.
\item A C++ project (\texttt{CMakeLists.txt} plus several
  \textbf{commented} cpp program files).
\end{itemize}


\section{Digital disk}

Let $I \subset \mathbb{Z}^2$ be a rectangular digital image. Let $Z \subset I$ be 
a digital set and $\bar{Z} = I \setminus Z$ be its complement set.

A digital set $Z$ is a \emph{digital disk} if and only if there exists a circle of 
center $\omega \in \mathbb{R}^2$ and of radius $\rho \in \mathbb{R}$ such that:  
\begin{equation}
  \left\{
  \begin{array}{l}
    \forall z \in Z, \: \| z - \omega \|^2 \leq \rho^2 \\
    \forall z \in \bar{Z}, \: \|  z - \omega \|^2 \geq \rho^2 \\
  \end{array}
  \right.
\end{equation}

%question lien disc de Gauss
\begin{qu}
Let us consider the Gauss digitization of a convex shape $X \subset \mathbb{R}^2$ at gridstep $h$:
\begin{displaymath}
  Dig(X,h) = X\cap (h\cdot \mathbb{Z}^2).
\end{displaymath}
Show that if $X$ is an Euclidean disk, then $Dig(X,h)$ is a digital disk. Show however that 
the converse is not true. 
\end{qu}

%question separation et determinant
\begin{qu}
There exists a unique circle passing by three digital points in general position. Show that we can test whether another digital
point lies INSIDE, ON or OUTSIDE such a circle with integer-only computations and without explicitly computing 
its center and radius. You may have a look at ``in-circle test'', broadly used in computational geometry, e.g. 
to compute the Delaunay triangulation of a point set.  
\end{qu}

\begin{qu}
Implement a function that checks whether two given digital sets are separated by a given circle passing by three
digital points or not.  
\end{qu}

\section{Convex programming}

Let us now consider algorithm \ref{algo:main}. 
It is a randomized and recursive algorithm that checks whether two point sets
 are separable by a circle in expected linear-time (see \cite{Welzl1991} for instance). 

Instead of having two point sets, we consider below only one set of points, denoted by $S$. 
Points of $S$ are either marked as $-$ (inner points) or marked as $+$ (outer points). 
  
The idea of algorithm \ref{algo:main} consists in removing a random point $p$ from $S$ 
and recursively compute a separating circle of the remaining points. 
\begin{itemize}
\item if $p$ is an inner points (resp. outer point) and it is located INSIDE (resp. OUTSIDE) or ON 
the returned separating circle $C$, ie. $C$ is separating for $S \cup \{p\}$, there is nothing else to do.
\item otherwise, 
 \begin{enumerate}
   \item either the two input sets are not circularly separable at all: $C$ is undefined.   
   \item or there exists a separating circle passing by $p$, which is recursively computed.  
 \end{enumerate}
\end{itemize}


\begin{algorithm}[Hhtbp]
  \KwIn{$S$, the set of inner and outer points, \\
  $B$, the set of points lying on the boundary of the separating circle }
  \KwResult{a separating circle $C$ (which may be undefined if the two sets are not circularly separable)}
  %
  \If{ $S$ is empty or $B$ contains three points } {
    $C \leftarrow $ smallest circle passing by the points of $B$ \;   
  } \Else {
    choose random $p \in S$ \; 
    $C \leftarrow $ separatingCircle($S - \{p\}$, $B$) \; 
    \If{ $C$ is defined and is not separating for $S$ } {
      $C \leftarrow $ separatingCircle($S - \{p\}$, $B \cup \{p\}$) \;  
    } 
  }
  \Return $C$ \; 
  \caption{separatingCircle($S$, $B$)}
  \label{algo:main}
\end{algorithm}


%% \begin{qu}
%% When its last parameter is $0$, what does algorithm \ref{algo:rec} ? Compare to the previous question.
%% When its last parameter is $k \in \{1,2,3\}$, what does algorithm \ref{algo:rec} ? 
%% \end{qu}

\begin{qu}
Is there always a unique separating circle for any $S$ ? Discuss possible cases, especially degenerate ones. 
\end{qu}


\begin{qu}
Implements a class that represents a smallest circle (possibly undefined) passing by the points of a given set.  
\end{qu}

%question implementation avec et sans randomization
\begin{qu}
Implement algorithm \ref{algo:main}. Provide test files. 
\end{qu}

\section{Experiments}

\begin{qu}
In order to check whether two connected digital sets $Z$ and $\bar{Z}$ are circularly separable, 
it is enough to consider only the digital boundaries of $Z$ and $\bar{Z}$. Implement a function
that takes as input the common contour of $Z$ and $\bar{Z}$ and that checks whether $Z$ and $\bar{Z}$
are circularly separable or not. You may use \textsc{DGtal} and more precisely the class {\tt GridCurve}
that can return the set of boundary digital points (see e.g. {\tt IncidentPointsRange}).    
\end{qu}

\begin{qu}
  Perform a running time analysis of your recognition function.  
  \begin{itemize}
  \item Implement a function that constructs the contour of a disk of a given radius.  
  \item Output the running time of your recognition function for disks of increasing radius. 
  \item Plot the graph of the running times against the size of the contour.
  \item Do you observe the expected linear-time complexity ?
  \end{itemize}	 
\end{qu}	

\section{Extra works}

\begin{qu}
Modify your recognition procedure in order to have an on-line algorithm, 
which takes input points two by two (one belonging to the boundary of $Z$ and one 
belonging to the boundary of $\bar{Z}$)
 and updates the current separating circle on the fly.
What is the time complexity of your procedure ?
\end{qu}

\begin{qu}
Use your on-line procedure to partition a contour into digital circular arcs and
 to compute the whole set of maximal digital circular arcs.  
\end{qu}

\nocite{Welzl1991}
\bibliographystyle{alpha}
\bibliography{refs}


\end{document}
