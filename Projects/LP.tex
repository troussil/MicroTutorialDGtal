\documentclass[a4paper, 11pt]{article}

\voffset -0cm
\hoffset 0.0cm
\textheight 23cm
\textwidth 16cm
\topmargin 0.0cm
\oddsidemargin 0.0cm
\evensidemargin 0.0cm

\usepackage{epsfig}  
\usepackage{setspace}
\usepackage{fancyheadings}
\usepackage{amsmath}
\usepackage{amssymb}
\usepackage{graphicx}
\usepackage{url}

%algo
\usepackage[english, linesnumbered, ruled, vlined]{algorithm2e}

\title{}
\author{}
\date{}

\newtheorem{qu}{Question}

\begin{document}

\begin{center}
	\LARGE \textbf{Project: ``Recognition of digital circle''}
\end{center}

\section*{Introduction}

The objective of this project is to implement an linear programming based algorithm
for the recognition of digital circles. 

We expect from you:
\begin{itemize}
\item A short report with answers to the ``formal'' questions and a
  description of your implementation choices and results.
\item A C++ project (\texttt{CMakeLists.txt} plus several
  \textbf{commented} cpp program files).
\end{itemize}


\section{Digital disk}

Let $I \subset \mathbb{Z}^2$ be a rectangular digital image. Let $Z \subset I$ be 
a digital set and $\bar{Z} = I \setminus Z$ be its complement set.

A digital set $Z$ is a \emph{digital disk} if and only if there exists a circle of 
center $\omega \in \mathbb{R}^2$ and of radius $\rho \in \mathbb{R}$ such that:  
\begin{equation}
  \left\{
  \begin{array}{l}
    \forall z \in Z, \: \| z - \omega \|^2 \leq \rho^2 \\
    \forall z \in \bar{Z}, \: \|  z - \omega \|^2 \geq \rho^2 \\
  \end{array}
  \right.
\end{equation}

%question lien disc de Gauss
\begin{qu}
Let us consider the Gauss digitization of a convex shape $X \subset \mathbb{R}^2$ at gridstep $h$:
\begin{displaymath}
  Dig(X,h) = X\cap (h\cdot \mathbb{Z}^2).
\end{displaymath}
Show that if $X$ is an Euclidean disk, then $Dig(X,h)$ is a digital disk. Show however that 
the converse is not true. 
\end{qu}

%question separation et determinant
\begin{qu}
There exists a unique circle passing by three digital points. Show that we can test whether another digital
point lies inside, on or outside such a circle with integer-only computations and without explicitely computing 
its center and radius. You may have a look at ``in-circle test'', broadly used in computational geometry, e.g. 
to compute the Delaunay triangulation of a point set.  
\end{qu}

\begin{qu}
Implement a function that checks whether two given digital sets are separated by a given circle passing by three
digital points or not.  
\end{qu}

Let us now consider algorithm \ref{algo:main} (which uses routine \ref{algo:rec}). 
It is a randomized and recursive algorithm that checks whether two point sets, 
denoted by $S^-$ and $S^+$, are separable by a circle in expected linear-time.  
This algorithm is based on the following idea:
for each point $s_i \in S^- \cup S^+$, if it belongs to $S^-$ (resp. $S^+$), 
but is located ouside (resp. inside) the current separating circle, then either it is on another
separating circle or the two input sets are not circularly separable at all.   
In the aim of deciding between these two alternatives, the set of possibly separating 
circles is restricted to circles passing by $s_i$ and the same algorithm is recursively called. 
A each call, the set of possibly separating circles is restricted so that the base case involves 
a unique circle passing by three given points and consists in checking whether it separates 
$S^-$ and $S^+$ or not.  

\begin{qu}
When its last parameter is $0$, what does algorithm \ref{algo:rec} ? Compare to the previous question.
When its last parameter is $k \in \{1,2,3\}$, what does algorithm \ref{algo:rec} ? 
\end{qu}

%question implementation avec et sans randomization
\begin{qu}
Implement algorithms \ref{algo:main} and \ref{algo:rec}. 
\end{qu}


%% \begin{algorithm}[Hhtbp]
%%   \KwIn{$S^-, S^+ \subset \mathbb{Z}^2$, the inner and outer point sets}
%%   \KwResult{``true'' if $S$ and $T$ are circularly separable, ``false'' otherwise}
%%   \KwOut{$a, b, c \in \mathbb{Z}^2$, three digital points caracterizing a separating circle if ``true''}
%%   %
%%   \tcp{initialisation}
%%   Construct the set of $S = S^- \cup S^+$ \; 
%%   \tcp{points of $S$ are numbered from $0$ to $|S|-1$, $|S|$ is the size of the set}
%%   Randomly permute the points of $S$ \;
%%   %cas particulier
%%   Initialize $a, b, c$ with three points of $S$ \tcp*{we assume that $|S| > 3$} 
%%   %
%%   \tcp{main loop}
%%   $i \leftarrow 3$ \; 
%%   areSeparable $\leftarrow$ TRUE \; 
%%   \While { \emph{areSeparable} and $i < |S|$ }{
%%     %
%%     \If{($s_i \in S^-$ and $s_i$ is outside the circle passing by $a,b,c$) \\ 
%%       or ($s_i \in S^+$ and $s_i$ is inside the circle passing by $a,b,c$) }{
%%       %
%%       $a \leftarrow s_i$ \; 
%%       areSeparable $\leftarrow$ areCircularlySeparableWithPoint($S^-$, $S^+$, $S$, $a$, $b$, $c$) \; 
%%       %
%%     }
%%     \Return areSeparable \; 
%%   }
%%   %
%%   \caption{areCircularlySeparable($S^-$, $S^+$, $a$, $b$, $c$)}
%%   \label{algo:3}
%% \end{algorithm}

%% \begin{algorithm}[Hhtbp]
%%   \KwIn{$S^-, S^+ \subset \mathbb{Z}^2$, the inner and outer point sets}
%%   \KwResult{``true'' if $S$ and $T$ are circularly separable, ``false'' otherwise}
%%   \KwOut{$a, b, c \in \mathbb{Z}^2$, three digital points caracterizing a separating circle if ``true''}
%%   %
%%   Initialize $b$ and $c$ with two points of $S$ \;  
%%   %
%%   \tcp{main loop}
%%   $i \leftarrow 3$ \; 
%%   areSeparable $\leftarrow$ TRUE \; 
%%   \While { \emph{areSeparable} and $i < |S|$ }{
%%     %
%%     \If{($s_i \in S^-$ and $s_i$ is outside the circle passing by $a,b,c$) \\ 
%%       or ($s_i \in S^+$ and $s_i$ is inside the circle passing by $a,b,c$) }{
%%       %
%%       $a \leftarrow s_i$ \; 
%%       areSeparable $\leftarrow$ areCircularlySeparableWithPoint($S^-$, $S^+$, $S$, $a$, $b$, $c$) \; 
%%       %
%%     }
%%     \Return areSeparable \; 
%%   }
%%   %
%%   \caption{areCircularlySeparableWithPoint($S^-$, $S^+$, $S$, $a$, $b$, $c$)}
%%   \label{algo:2}
%% \end{algorithm}

\begin{algorithm}[Hhtbp]
  \KwIn{$S^-, S^+ \subset \mathbb{Z}^2$, the inner and outer point sets \\
  $p_1, p_2, p_3 \in \mathbb{Z}^2$, three points characterizing a circle }
  \KwResult{``true'' if $S^-$ and $S^+$ are circularly separable, ``false'' otherwise}
  \KwOut{$p_1, p_2, p_3$, three points caracterizing a separating circle if ``true''}
  %
  \tcp{initialisation step}
  Construct the set of $S = S^- \cup S^+$ \; 
  Randomly permute the points of $S$ \;
  \tcp{points of $S$ are numbered from $0$ to $|S|-1$, $|S|$ is the size of the set}
  Initialize $p_1, p_2, p_3$ with three points of $S$ \tcp*{we assume here that $|S| > 3$}
  \tcp{recursive step}
  \Return areCircularlySeparable($S^-$, $S^+$, $S$, $p_1$, $p_2$, $p_3$, $3$) \; 
  %
  \caption{areCircularlySeparable($S^-$, $S^+$, $p_1$, $p_2$, $p_3$)}
  \label{algo:main}
\end{algorithm}

\begin{algorithm}[Hhtbp]
  \KwIn{$S^-, S^+ \subset \mathbb{Z}^2$, the inner and outer point sets, $S = S^- \cup S^+$ \\
  $p_1, p_2, p_3 \in \mathbb{Z}^2$, three points characterizing a circle \\
  $n$, degree of freedom: only the first $p_k$ for $k$ from $1$ to $n$ are variable}
  \KwResult{``true'' if $S^-$ and $S^+$ are circularly separable, ``false'' otherwise}
  \KwOut{$p_1, p_2, p_3$, three points caracterizing a separating circle if ``true''}
  %
  \If{$n \geq 0$}{
    Initialize $p_k$ for $k$ from $1$ to $n$ with $n$ points of $S$ \;  
    %
    %\tcp{main loop}
    $i \leftarrow 0$ \; 
    areSeparable $\leftarrow$ TRUE \; 
    \While { \emph{areSeparable} and $i < |S|$ }{
      %
      \If{($s_i \in S^-$ and $s_i$ is outside the circle passing by $p_1, p_2, p_3$) \\ 
        or ($s_i \in S^+$ and $s_i$ is inside the circle passing by $p_1, p_2, p_3$) }{
        %
        $p_n \leftarrow s_i$ \; 
        areSeparable $\leftarrow$ areCircularlySeparable($S^-$, $S^+$, $S$, $p_1$, $p_2$, $p_3$, $n+1$) \; 
        %
      }
      \Return areSeparable \; 
    }    
  }
  %
  \Else{
    \Return FALSE \; 
  }
  %
  \caption{areCircularlySeparable($S^-$, $S^+$, $S$, $p_1$, $p_2$, $p_3$, $n$)}
  \label{algo:rec}
\end{algorithm}

\section{Digital circle}

Let us consider now the contour $C$ of a digital set $Z$. If $Z$ is connected
and does not have any hole, its contour consists of a circular sequence of 1-cells. 
Each 1-cell of the contour is the common edge of a 2-cell whose center is a point of $Z$ 
and a 2-cell whose center is a point of $\bar{Z}$. Let us denote by $C^-$ (resp. $C^+$)
the set of digital points of $Z$ (resp. $\bar{Z}$) that are the center of a 2-cell 
incident to a 1-cell of $C$.  
  
A contour $C$ is a \emph{digital circle} if and only if there exists a circle of 
center $\omega \in \mathbb{R}^2$ and of radius $\rho \in \mathbb{R}$ such that:  
\begin{equation}
  \left\{
  \begin{array}{l}
    \forall z \in C^-, \: \| z - \omega \|^2 \leq \rho^2 \\
    \forall z \in C^+, \: \| z - \omega \|^2 \geq \rho^2 \\
  \end{array}
  \right.
\end{equation}

\begin{qu}
Implement a function that checks whether $C$ is a digital circle or not. 
\end{qu}

\begin{qu}
  Perform a running time analysis of your recognition function.  
  \begin{itemize}
  \item Implement a function that constructs circles of a given radius.  
  \item Output the running time of your recognition function for disks of increasing radius. 
  \item Plot the graph of the running times against the size of the contour.
  \item Do you observe the expected linear-time complexity ?
  \end{itemize}	 
\end{qu}	

\section{Extra works}

\begin{qu}
Modify your recognition procedure in order to have an on-line algorithm, 
which takes input points two by two (one belonging to $C^-$ and one belonging to $C^+$)
 and updates the current separating circle on the fly.
What is the time complexity of your algorithm ?
\end{qu}

\begin{qu}
Use this algorithm to segment a contour into digital circular arcs or to compute
the whole set of maximal digital circular arcs.  
\end{qu}



\end{document}
